\documentclass{jsarticle}
\usepackage{amsmath}
\title{再生産関係の自己相関推定におけるジャックナイフ解析}
\author{西嶋 翔太\thanks{水産研究・教育機構 水産資源研究所}}
\date{2021年6月7日}
 
\begin{document}
 
\maketitle

加入量$R_t$と親魚量$SSB_t$ ($t = 1,2,...,T$) データにおいて、$t = s$のサンプルを除いたときの、1次の自己相関構造 (AR(1)) を持つ場合の再生産関係の推定について考える。再生産関係からの加入量の残差を
$$ \epsilon_t = log(R_t) - log(f(a,b,SSB_t))$$
と定義する。ここで、 $f(a,b,SSB_t)$は再生産関係からの予測値を表す。
このとき、予測される残差は
\begin{equation}
E({\epsilon}_t) = 
\begin{cases}
0 & (t = 1) \\
\rho \epsilon_{t-1} & (otherwise)
\end{cases}
\end{equation}
で表される。$\rho$は1次の自己相関係数である ($-1 < \rho < 1$)。
しかし、ジャックナイフ解析では$t = s$のサンプルが利用できないため、 $\epsilon_{s}$が不明である。
したがって、$t = s + 1$のときの残差は$t = s - 1$からの予測値となり、
\begin{equation}
E({\epsilon}_t) = 
\begin{cases}
0 & (t = 1) \\
\rho^2 \epsilon_{t-2} & (t = s + 1) \\
\rho \epsilon_{t-1} & (otherwise)
\end{cases}
\end{equation}
となる。  

AR(1) のときの分散は
\begin{equation}
V({\epsilon}_t) = 
\begin{cases}
\dfrac{\sigma ^2}{1 - \rho ^2} & (t = 1) \\
\sigma ^2 & (otherwise)
\end{cases}
\end{equation}
となり、ここで$\sigma$は$t \geq 1$のときの標準偏差である。しかし、ジャックナイフ解析において$t = s + 1$のときの残差は$t = s - 1$からの予測値となるため、分散は
\begin{equation}
V({\epsilon}_t) = 
\begin{cases}
\dfrac{\sigma ^2}{1 - \rho ^2} & (t = 1) \\
(1 + \rho ^2) \sigma ^2 & (t = s + 1) \\
\sigma ^2 & (otherwise)
\end{cases}
\end{equation}
となる。
したがって、尤度は
\begin{equation}
{\epsilon}_t \sim 
\begin{cases}
Normal \left( 0, \dfrac{\sigma ^2}{1 - \rho ^2} \right) & (t = 1) \\
Normal \left(\rho^2 \epsilon_{t-2}, (1 + \rho ^2) \sigma ^2 \right) & (t = s + 1) \\
Normal \left(\rho \epsilon_{t-1}, \sigma ^2 \right) & (t \neq 1, s, s+1)
\end{cases}
\end{equation}
で与えられる。
重み付け最小二乗法として、
$$ \xi _t = \epsilon _t - E( \epsilon_t ) $$
と置くと、標準偏差は
\begin{equation}
\sigma = \sqrt{ \frac{\left( 1 - \rho ^ 2 \right) \xi _{1} ^ 2 + \xi _{2} ^ 2 + ... + \xi _{s-1} ^ 2 + \tfrac{1}{(1 + \rho ^2)} \xi _{s+1} ^ 2 + ... + \xi _{T} ^ 2}{T - 1}} 
\end{equation}
で求めることができる。

\end{document}
